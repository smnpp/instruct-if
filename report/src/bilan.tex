\section*{Conclusion}
\addcontentsline{toc}{section}{Bilan}
Ce TP nous a permis de prendre en main pour la première fois le logiciel ANSYS WorkBench. Il nous a permis de concevoir une éprouvette et de simuler un effort en traction et ainsi étudier son comportement.
\newline
\\
Nous avons pu comparer deux méthodes de calculs : la méthode des éléments finis et celle de la résistance des matériaux. Nous avons donc pu voir les limites de la RDM comme notamment l'imprécision du résultat aux extrémités de l'éprouvette pour des contraintes d'effort mais également pour les déformations. 
Cependant la méthode de la RDM reste précise le long de la fibre neutre au centre de l'éprouvette.
\newline
\\
De plus, le logiciel ANSYS WorkBench nous a permis d'étudier de manière précise le lieu de contrainte maximale avec notamment une réduction du maillage de manière localisée.